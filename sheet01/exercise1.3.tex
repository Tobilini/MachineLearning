\documentclass{article}
\usepackage{graphicx} % Required for inserting images
\usepackage{amsmath}

\title{Exercise 1.3}
%\author{Muhammad Shaafay Saqib and Raaez}
\date{\today}
\begin{document}
\maketitle

\section{Formulate mathematically a loss function}

   \[
   \sum_{i < j} {1}(y_i > y_j)
   \]

   Boolean statements will evaluate to either 1 (True) or 0 (False). This loss function checks that each input is higher ranked than the following. Any correct rankings do not contribute to the loss of value, whereas incorrect ones do.


\section{Which model is better at
ranking? Why is the squared error problematic in this case?}

The tuples that meet the condition $i < j$ are $(1,2)$, $(1,3)$ and $(2,3)$, so these will be used in the summation.

For the first model $\hat{y}_1$, the loss function evaluates to the following.

   \begin{multline}
   \\
   {1}(1 > 3) + {1}(1 > 2) + {1}(3 > 2) \\
   = 0 + 0 + 1 \\
   = 0
   \end{multline}

For the second model $\hat{y}_2$, the loss function evaluates to the following.

   \begin{multline}
   \\
   {1}(2 > 3) + {1}(2 > 7) + {1}(3 > 7) \\
   = 0 + 0 + 0 \\
   = 0
   \end{multline}

So the second model $\hat{y}_2$ is better. Squared error is problematic because it would penalise $\hat{y}_2$ more for having a larger range despite it having a more correct ranking.

\end{document}
